\documentclass[a4paper, 10pt]{article}
\usepackage[T1]{fontenc}
\usepackage[utf8]{inputenc}
\usepackage{graphicx}
\usepackage{xcolor}


\usepackage[colorlinks=true]{hyperref}
\usepackage{tabularx}

\usepackage{amsmath,amssymb,amsthm,textcomp}
\usepackage{enumerate}
\usepackage{multicol}
\usepackage{tikz}
\usepackage[english, russian]{babel}

\usepackage{cases}

\usepackage{geometry}
\geometry{total={210mm,297mm},
	left=15mm,right=15mm,%
	bindingoffset=0mm, top=20mm,bottom=20mm}

\usepackage{setspace}





\newcommand{\R}{\mathbb{R}} 
\newcommand{\Z}{\mathbb{Z}}
\newcommand{\x}{\times}
\newcommand{\N}{\mathbb{N}} 
\newcommand{\Q}{\mathbb{Q}} 
\newcommand{\Co}{\mathbb{C}}
\newcommand{\F}{\mathbb{F}}
\newcommand{\al}{\alpha}
\newcommand{\GCD}{\text{НОД}}
\newcommand{\D}{\displaystyle} 


\title{
	Тестирование по математическому анализу.
}
\author{Кусакин Александр Александрович, ПИ-176.
}
\date{}


\begin{document}
\maketitle
\begin{spacing}{1}
%%%%%%%%%%%%%%%%%%%%%%%%%%%%%%%%%%%%%%%%%%%%%%%%%%%%%%%%%%%%
%%%%%%%%%%%%%%%%%%%%%%%%%%%%%%%%%%%%%%%%%%%%%%%%%%%%%%%%%%%%
%%%%%%%%%%%%%%%%%%%%%%%%%%%%%%%%%%%%%%%%%%%%%%%%%%%%%%%%%%%%

	
\begin{center}
	\fbox{Задание 1.}
\end{center}
		
\noindent \textit{Найдите номер члена последовательности $\displaystyle y_n = \frac{3n+191}{3n+2}$, наиболее близкое к числу $2$.
}\\
\noindent \textbf{Решение.} Выделим целую часть. $\D y_n = 1 + \frac{189}{3n+2}$. Заметим, что $y_n$ наиболее близко к числу 2, тогда и только тогда когда $\D \frac{189}{3n + 2}$ близко к числу 1. Решим уравнение $\D 189 = 3n+2 \Leftrightarrow n = \frac{187}{3} \approx 62.(3)$. Рассмотрим случаи:
\begin{enumerate}
	\item При n = 62. $\dfrac{3*62+191}{3*62+2} = 2.005319$
	\item При n = 63. $\dfrac{3*63+191}{3*63+2} = 2.021276$
\end{enumerate}
Остальные значения n рассматривать не имеет смысла.\\
\noindent \textbf{Ответ}: 62.

\begin{center}
	\fbox{Задание 2.}
\end{center}

\noindent \textit{Решите уравнение, если известно, что $|x|<1$: $2x+1+x^2-x^3+x^4-x^5+...=\dfrac{13}{6}$}

\noindent \textbf{Решение.} Заметим, что $x^2-x^3+x^4-x^5+...$ является бесконечно убывающей геометрической прогрессией, где $b_1 = x^2$, $q = -x$. Таким образом, сумма равна $S = \dfrac{x^2}{1+x}$. \\
Заменим в имеющемся уравнении $2x+1+x^2-x^3+x^4-x^5+...=\dfrac{13}{6}$ прогрессию на её сумму и получим уравнение в следующем виде: $2x+1 + \dfrac{x^2}{1+x} = \dfrac{13}{6} \Leftrightarrow 2x + \dfrac{x^2}{1+x} = \dfrac{7}{6} \Leftrightarrow 12x(1+x) + 6x^2 = 7(1+x) \Leftrightarrow 12x + 12x^2 + 6x^2 = 7+ 7x \Leftrightarrow 18x^2+5x-7=0 \Rightarrow \left[\begin{array}{l}
	x = -\dfrac{7}{9}\\
	x = \dfrac{1}{2}
\end{array}\right.$\\
\textbf{Ответ:} $\{-\dfrac{7}{9}, \dfrac{1}{2}\}$

\begin{center}
	\fbox{Задание 3.}
\end{center}

\noindent \textit{ Вычислите $\lim\limits_{x \to 0,5}\dfrac{arccosx+\pi sinx}{\pi cos\pi x+2arcsinx}$
}

\noindent \textbf{Решение.} Попробуем подставить вместо x 0,5 (значение к которому стремится x). Получим следующее $\dfrac{\dfrac{\pi}{3}+\pi}{0+\dfrac{\pi}{3}} = \dfrac{\dfrac{4\pi}{3}}{\dfrac{\pi}{3}}=\textbf{4}.$\\
\textbf{Ответ:} 4

\begin{center}
	\fbox{Задание 4.}
\end{center}

\noindent \textit{Воспользовавшись определением, найдите производную функции в точке x: y=$\dfrac{1}{x^2}$
}

\noindent \textbf{Решение.} Сначала вспомним определение производной в точке x: $f'(x)=\lim\limits_{\Delta x \to 0}\dfrac{\Delta f}{\Delta x}=\lim\limits_{\Delta x \to 0}\dfrac{f(x+\Delta x)-f(x)}{\Delta x}$. Подставим в эту формулу наше условие и получим следующее: $\lim\limits_{\Delta x \to 0} \dfrac{\dfrac{1}{(x+\Delta x)^2}-\dfrac{1}{x^2}}{\Delta x}=\lim\limits_{\Delta x\to 0}\dfrac{\dfrac{1}{x^2+2x\Delta x+\Delta x^2}-\dfrac{1}{x^2}}{\Delta x}=\lim\limits_{\Delta x\to 0}\dfrac{-2x\Delta x-\Delta x^2}{x^2(x+\Delta x)^2\Delta x}=\lim\limits_{\Delta x\to 0}-\dfrac{2x+\Delta x}{x^2(x+\Delta x)^2}=-\dfrac{2}{x^3}$\\
\textbf{Ответ:} $-\dfrac{2}{x^3}$

\begin{center}
	\fbox{Задание 5.}
\end{center}

\noindent \textit{При каких значениях параметра a касательные к графику функции y=$4x^2-|a|x$, проведенные в точках его пересечения с осью x, образуют между собой угол 60\textdegree
}

\noindent \textbf{Решение.} \\
\textbf{Ответ:} $\pm 1$
		
\end{spacing}
\end{document}